\documentclass[10pt,a4paper]{article}
\usepackage[paper=a4paper]{geometry}

\usepackage[utf8x]{inputenc}
\usepackage[spanish]{babel}

\usepackage{mathtools}
\usepackage{amsmath}
\usepackage{amsfonts}
\usepackage{amssymb}

\usepackage{xcolor}
\usepackage{listingsutf8}
\usepackage{booktabs}
\usepackage{hyperref}
\usepackage{multirow}

\usepackage{caption}
\usepackage{subcaption}

\usepackage{algorithm}
\usepackage[noend]{algpseudocode}

\usepackage{graphicx}
\usepackage{tikz}
\usepackage{relsize}
\usepackage{epstopdf}

\DeclarePairedDelimiter{\ceil}{\lceil}{\rceil}

% set the default code style
\lstset{
    frame=tb, % draw a frame at the top and bottom of the code block
    tabsize=4, % tab space width
    showstringspaces=false, % don't mark spaces in strings
    numbers=left, % display line numbers on the left
    commentstyle=\color{green}, % comment color
    keywordstyle=\color{blue}, % keyword color
    stringstyle=\color{red} % string color
}

% mathy stuff
\newtheorem{theorem}{Theorem}[section]
\newtheorem{lemma}[theorem]{Lemma}
\newtheorem{proposition}[theorem]{Proposición}
\newtheorem{corollary}[theorem]{Corollary}

\newenvironment{proof}[1][Demostración]{\begin{trivlist}
\item[\hskip \labelsep {\bfseries #1}]}{\end{trivlist}}
\newenvironment{definition}[1][Definición]{\begin{trivlist}
\item[\hskip \labelsep {\bfseries #1}]}{\end{trivlist}}
\newenvironment{example}[1][Example]{\begin{trivlist}
\item[\hskip \labelsep {\bfseries #1}]}{\end{trivlist}}
\newenvironment{remark}[1][Remark]{\begin{trivlist}
\item[\hskip \labelsep {\bfseries #1}]}{\end{trivlist}}

\newcommand{\qed}{\nobreak \ifvmode \relax \else
      \ifdim\lastskip<1.5em \hskip-\lastskip
      \hskip1.5em plus0em minus0.5em \fi \nobreak
      \vrule height0.75em width0.5em depth0.25em\fi}

\title{Aprendizaje Automatico \\ Trabajo Práctico 2 \\ QLearning }

\newcommand{\order}[1]{$\mathcal{O}(#1)$}

\begin{document}

%% cover page

\maketitle

\bigskip

\begin{table}[h]
\centering
\begin{tabular}{|l l l l|}
\hline
Integrante       & \multicolumn{1}{c}{LU}     & Correo electrónico       	& Carrera \\ \hline
Martin Baigorria & \multicolumn{1}{c}{575/14} & martinbaigorria@gmail.com & computación (licenciatura) \\ 
Damián Furman & \multicolumn{1}{c}{936/11}& damian.a.furman@gmail.com & computación (licenciatura)\\
Germán Abrevaya & \multicolumn{1}{c}{-} & germanabrevaya@gmail.com & física (doctorado)\\ \hline
\end{tabular}
\end{table}

\vfill

\begin{center}
\textbf{Reservado para la cátedra}
\end{center}
\begin{table}[h]
\centering
\begin{tabular}{|l|l|l|}
\hline
Instancia       & Docente & Nota \\ \hline
Primera entrega &         &      \\ \hline
Segunda entrega &         &      \\ \hline
\end{tabular}
\end{table}

\newpage
\tableofcontents
\newpage

% end cover page

\section{Introducción}

Hablar un poco de QLearning y cadenas de Markov.

Hablar un poco sobre la cantidad de estados posibles que hay en 4 en linea. La cadena de Markov asociada tiene 1 estado por posible configuracion del tablero, y una transicion por cada accion posible a otro estado.

¿Cuál es el espacio de estados?
¿cuán rápido se puede explorar?
¿cómo cambia la inicialización de Q con respecto a la velocidad de aprendizaje?
¿qué importancia tiene la temperatura y la velocidad con que se enfría el sistema si se decide usar la distribution Boltzmann?
¿qué efecto tiene cambiar la tasa de aprendizaje?


Implementar el alpha dinamico, a medida que pasa el tiempo alpha cae. Faltan cosas, pero la base del codigo esta.

Jugar con todos los parametros de los players, y el tablero.

\end{document}